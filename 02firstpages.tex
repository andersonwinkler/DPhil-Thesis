\pagestyle{empty}
% ======[COVER]======
\begin{center}
%\begin{small}
\textbf{\textsc{a thesis presented for the degree of}}

\textbf{\textsc{Doctor of Phylosophy}}
%\end{small}

\vspace*{\fill}

\begin{Huge}
{\color{oxblue}\textbf{Widening the applicability}}
\end{Huge}

\vspace{2mm}

\begin{Huge}
{\color{oxblue}\textbf{of permutation inference}}
\end{Huge}

\vspace*{\fill}

\textbf{Anderson M.\ Winkler}

\textbf{St.\ Edmund Hall}

\vspace*{\fill}

\includegraphics[scale=1.6]{figures/ox_logo_special_gold_pos.eps}

\vspace*{\fill}

\textbf{\textsc{fmrib} Analysis Group}

\textbf{Nuffield Department of Clinical Neurosciences}

\textbf{University of Oxford}

\vspace*{\fill}

\textbf{Trinity/2016}
\end{center}

% ======[ABSTRACT]======
\cleardoublepage
\setstretch{1}
\renewcommand\abstractname{{\Large Abstract}}
\begin{abstract}
Permutation tests can provide exact control of false positives under the reasonable assumption of exchangeability. There are, however, common examples in which global exchangeability does not hold, such as with repeated measurements or tests in which subjects are members of pedigrees. To allow permutation inference in such cases, we test the null hypothesis using only a subset of all otherwise possible permutations, thus retaining the original joint distribution unaltered. Here the notion of exchangeability blocks is extended, allowing nesting in a hierarchical, multi-level definition. We do not explicitly model the degree of dependence between observations, only the lack thereof; such dependence is implicitly accounted for by the hierarchy and by the permutation scheme. The strategy is compatible with heteroscedasticity and variance groups, and can be used with permutations, sign flippings, or both combined. Additionally, we exploit properties of test statistics to obtain accelerations irrespective of generic software or hardware improvements. We compare six different approaches using both synthetic and real data, assessing the methods in terms of their error rates, power, agreement with a reference result, and the risk of taking a different decision regarding the rejection of the null hypotheses (known as the resampling risk). Finally, we investigate the different methods for assessment of cortical volume and area from magnetic resonance images using surface-based methods, and show that volume mostly mirrors variability in surface area, while having little sensitivity to cortical thickness, and this remains the case even when volume is assessed using an improved analytic method. Using data from young adults born with very low birth weight and coetaneous controls, we show that instead of volume, the permutation-based Non-Parametric Combination (NPC) of thickness and area is a more sensitive option for studying joint effects on these two quantities, giving equal weight to variation in both, allowing a better characterisation of biological processes that can affect brain morphology.
\end{abstract}

% ======[ACKNOWLEDGEMENTS]======
\cleardoublepage
\newpage
%\setstretch{\lspac}
%\vspace*{\fill}
%\begin{center}
%\begin{Large}
%\textbf{Acknowledgements}
%\end{Large}
%\end{center}

\chapter*{Acknowledgements}
\setstretch{\lspac}

\noindent
\`{A} minha fam\'{i}lia.

\vspace{3mm}\noindent
To my thesis advisors, Prof.\ Dr.\ Stephen M.\ Smith and Prof.\ Dr.\ Thomas E.\ Nichols.

\vspace{3mm}\noindent
I am very much indebted to the National Resarch Council of Brazil (\textsc{cnp}q), without which this work would have been impossible.

\vspace{3mm}\noindent
I am thankful to the strong, prolific, and enriching collaboration that have helped and improved all the chapters. Chapter~\ref{sec:ptree} received helpful feedback from Matthew A.\ Webster and Diego Vidaurre. Likewise, Chapter~\ref{sec:accel} received much welcome input from Gerard R.\ Ridgway and Gwena\"{e}lle Douaud. Chapter~\ref{sec:cortex} would have been impossible without the close collaboration of Lars M.\ Rimol, Knut J.\ Bjuland, Jon Skranes, Douglas N.\ Greve, Asta K.\ H{\aa}berg, and Mert R.\ Sabuncu.

\vspace{3mm}\noindent
My time in Oxford was made even better thanks to the people in the colleges to which I was at one time or another affiliated: St.\ Cross and St.\ Edmund Hall. Under the risk unfair omissions, some I would like to thank nominally: Prof.\ Keith Gull, Dr.\ Richard Willden. Prof.\ Heidi Johansen--Berg, Prof.\ Paul M.\ Matthews, Dr.\ Charlotte Stagg, Sir Mark Jones, Prof.\ Petros Ligoxygakis, and Dr. Katie Warnaby, all of whom, through their remarkable work and friendliness, have made these years extremely enjoyable.

\vspace{3mm}\noindent
At \textsc{fmrib}, the list is long, and again, in the hope of not unfairly omitting anyone, I am much thankful to all the members of the Analysis Group not already mentioned above, and listed here in no particular order: Mois{\'e}s Hern{\'a}ndez--Fern{\'a}ndez, Fidel Alfaro--Almagro, Samuel J.\ Harrison, Jonathan Hadida, Giles Colclough, Zobair Arya, Emmanuel Vallee, Riham Satti, Eugene P.\ Duff, Matteo Bastiani, Evangelos Roussos, Michiel Cottaar, Eelke Visser, Stamatios Sotiropoulos, Mark Jenkinson, Jesper Andersson, Saad Jbabdi, Janine Bijsterbosch,  Oiwi Parker Jones, Sean Fitzgibbon, Paul McCarthy, Ludovica Griffanti, Emma Robinson, Jelena Bo\v{z}ek Mouthuy, David Flitney and Duncan Mortimer. Additionally, I would like to thank Susan Field for being tirelessly solicitous since even before the time I arrived.

\vspace{3mm}\noindent
Finally, I would like to thank the Sir Edward Penley Abraham Fund and St.\ Edmund Hall for the honour of being twice the recipient of the ``Cephalosporin Award'', providing the much needed further support during this period, and the Nuffield Department of Clinical Neurosciences and St.\ Edmund Hall for the funding that allowed attending and presenting work at conferences.
